\documentclass[letterpaper,twocolumn,10pt]{article}
\usepackage{usenix,epsfig,endnotes}
\usepackage[ngerman]{babel}

\usepackage[ngerman]{babel}
\usepackage[utf8]{inputenc}%	for umlauts
\usepackage[babel,german=guillemets]{csquotes}

\usepackage{blindtext}

\begin{document}

%don't want date printed
\date{7.12.16}

%make title bold and 14 pt font (Latex default is non-bold, 16 pt)
\title{\Large \bf IT-Sicherheitsgesetzt: Gesetzes zur Erhöhung der Sicherheit  
informationstechnischer Systeme\\ mit Fokus auf ...}

\author{
{\rm Alexander\ Lüdke}\\
MatrNr. 548965
\and
{\rm Nils Brandt}\\
MatrNr. 549906
}

\maketitle

\thispagestyle{empty}


\subsection*{Abstract}

\section{Einleitung}

\section{Gesetzgegenstand}
	\subsection{Problem und Ziel}
	\subsection{Entgegnung}

\section{Pro/Contra - Aussagen der Sachverständigen}

%Die Sachverständigen:
%Prof. Dr. Gerrit Hornung, Universität Passau, Lehrstuhl für öffentliches Recht, IT-Recht und Rechtsinformatik
%Linus Neumann, Chaos Computer Club (CCC), Berlin
%Iris Plöger, Bundesverband der Deutschen Industrie e. V., Leiterin der Abteilung Digitalisierung
%Prof. Dr. Alexander Roßnagel, Universität Kassel, Institut für Wirtschaftsrecht
%Prof. Dr.-Ing. Jochen Schiller, Freie Universität Berlin, Institute of Computer Science
%Dipl. Ing. (FH) Thomas Tschersich, Deutsche Telekom AG, Leiter Group Security Services
%Dr. Axel Wehling, Gesamtverband der Deutschen Versicherungswirtschaft e. V., Mitglied der Hauptgeschäftsführung, Geschäftsführer des Krisenreaktionszentrums der deutschen Versicherungswirtschaft

{\footnotesize \bibliographystyle{acm}\bibliography{plain}}

\theendnotes % footnotes

\end{document}







