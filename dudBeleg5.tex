% /===========================================================================\
%
%   preamble
%
% \===========================================================================/

\documentclass[a4paper,letterpaper,twocolumn,10pt,ngerman]{article}

% /=================================================================\
%   	standard
% \=================================================================/
\usepackage{babel}
\usepackage[utf8]{inputenc}	% using umlauts = encoding
\usepackage[T1]{fontenc}

\usepackage{mdwlist}	% to shrink the spaces between the items into a list
\usepackage{blindtext}	% fill parts with blindtex
\usepackage{verbatim}	%u sing in multiline comments

\usepackage[german=guillemets]{csquotes}
\usepackage[
			backend=biber,
			style=alphabetic-verb,
			citestyle=alphabetic-verb
			]
			{biblatex}	% Einbinden von biblatex zur Literaturverwaltung.
\addbibresource{literaturelibrary/litLib.bib}	% Literatursammlung


% /=================================================================\
%   	commands
% \=================================================================/
\newcommand{\descitem}[1]{\textbf{#1}}	% style of description items

% /===========================================================================\
%
%   init document
%
% \===========================================================================/

\begin{document}

% /===========================================================================\
%
%   title
%
% \===========================================================================/

% don't want date printed
\date{}

% make title bold and 14 pt font (Latex default is non-bold, 16 pt)
\title{\Large \bf Die öffentliche Anhörung des Innenausschusses zum Thema "`Gesetz zur Erhöhung der Sicherheit informationstechnischer Systeme (IT-Sicherheitsgesetzt)"'}

\author{
{\rm Alexander\ Lüdke}\\
MatrNr. 548965
\and
{\rm Nils Brandt}\\
MatrNr. 549906
}

\maketitle

\thispagestyle{empty}

% /===========================================================================\
%
%   main part
%
% \===========================================================================/

\section*{Abstract}
\label{sec:Abstract}
Das folgende Dokument ist als Beleg anzusehen, welches für den Studiengang Angewandte Informatik im Fach Datenschutz und Datensicherheit erstellt wurde. Es befasst sich hauptsächlich mit dem Inhalten der öffentlichen Anhörung des deutschen Bundestages zum Thema "`Gesetz zur Erhöhung der Sicherheit informationstechnischer Systeme"'  und der damit einhergehenden Argumentation der Sachverständige. Da der Umfang dieser Ausarbeitung die zwei Seiten nicht überschreiten darf, nehmen die Autoren einen Teilaspekt der Anhörung und werden diesen, durch ausgiebige Recherchen, näher beleuchten. 

\section{Einleitung}
\label{sec:Einleitung} 
% Welche Mängel weisst das Gesetzt auf und wie kann ihm entgegengewirkt werden?
% Wie sehen die Agumente der Gegner aus und wie wird Ihnen begegnet?
% Haben Befürworter Zweifel?

\section{Vom Entwurf zum Gesetz}
\label{sec:EntwurfzumGesetz}
Gegenstand der öffentlichen Anhörung  war der von der Bundesregierung vorgelegte Gesetzentwurf \cite{GesEntw15}, der das IT~-Sicherheitsgesetzt anpassen bzw. erweitern sollte und dabei die Erhöhung der Sicherheit informationstechnischer Systeme zum Gegenstand hatte. 
\begin{quotation}
"`Die vorgesehenen Neuregelungen dienen dazu, [...] um  den  aktuellen  und zukünftigen  Gefährdungen  der IT-Sicherheit wirksam begegnen zu können. Ziel des Gesetzes sind die Verbesserung der IT~-Sicherheit von Unternehmen, der verstärkte Schutz der Bürgerinnen und Bürger im Internet und in diesem Zusammenhang auch die Stärkung von BSI und Bundeskriminalamt (BKA)."' \cite[S. 1]{GesEntw15} 
\end{quotation}

\subsection{Standpukte der Sachverständigen}
\label{subsec:StandpunkteSachverständige}
Um nahezu alle Aspekte dieses Sachverhaltes zu betrachten, wurde im Vorfeld Stellungnahme von unterschiedlichen Personen -- bzw. Interessensgruppen eingeholt, die während der Anhörung die Möglichkeit erhielten, diese näher zu erläutern. Die geladenen Sachverständigen waren folgende

\begin{description*}
    \item	{\descitem{Iris Plöger}}, Bundesverband der Deutschen Industrie e.V., Leiterin der Abteilung Digitalisierung
    \item	{\descitem{Dipl.-Ing.  (FH) Thomas Tschersich}}, Deutsche Telekom AG, Leiter Group Security Services
    \item	{\descitem{Dr. Axel Wehling}}, Gesamtverband der Deutschen Versicherungswirtschaft e.V.
    \item	{\descitem{Linus Neumann}}, Chaos Computer Club (CCC), Berlin

	\item	{\descitem{Michael Hange}}, Präsident des Bundesamtes für Sicherheit in der Informationstechnik
    \item	{\descitem{Prof. Dr. Gerrit Hornung}}, Universität Passau, Lehrstuhl für öffentliches Recht, IT-Recht und Rechtsinformatik
    \item	{\descitem{Prof. Dr. Alexander Roßnagel}}, Universität Kassel, Institut für Wirtschaftsrecht
    \item	{\descitem{Prof. Dr.-Ing. Jochen Schiller}}, Freie Universität Berlin, Institute of Computer Science
\end{description*}

Aus den Schilderungen der jeweiligen Sachverständigen traten u.a. die folgenden Punkte in den Vordergrund. Die Stellung der BSI als zentrale Meldestelle. Auf Grund ihrer Zugehörigkeit zum BMI und der damit einhergehenden evtl. direkten Zusammenarbeit mit dem BND tritt hier ein gewisses Misstrauen in den Vordergrund. Wie in der Stellungnahme von Herrn Neumann \cite[vlg.][S. 4]{NeuCCC15}  beschreiben, sollte das BSI das Monitoring von kritischen Angriffszielen übernehmen, wäre es eine rein defensive Ausrichtung, was jedoch im Gegensatz zu den offensiven Ambitionen des BMI stehen würde. Hinzu kommt, dass das BSI nach §7a Abs.2 ITSG die aus den Untersuchungen gewonnenen Erkenntnisse nur zur Erfüllung der Aufgaben verwenden darf. Das Bundesamt darf seine Erkenntnisse weitergeben und veröffentlichen, soweit dies zur Erfüllung dieser Aufgaben erforderlich ist. Zuvor ist dem Hersteller der betroffenen Produkte und Systeme mit angemessener Frist Gelegenheit zur Stellungnahme zu geben. Wie bereites durch die Redaktion von Netzpolitik.org festgestellt wurde \cite{Bis15}, liegt die Anwendung des Gesetzestextes aufgrund schwammiger Formulierungen und industriefreundlicher Meldepflichten im Ermessensspielraum der Wirtschaft. 

Kritik:
- Fokus liegt auf kritische Infrastrukturen und weniger auf Wirtschaft und Privatanwender
- Hoher bürokratischer Bearbeitungsaufwand

\section{Résumé}
\label{sec:Resume}

\newpage

% /===========================================================================\
%
% 	bibliography
%
% \===========================================================================/
\printbibliography

% /=================================================================\
%   	notice
% /=================================================================/
\begin{comment}	
	timeline:
		20.04.2015 	öffentlichen Anhörung des Innenausschusses
		https://www.bundestag.de/dokumente/textarchiv/2015/kw17_pa_inneres/367474
		12.06.2015	Bun­des­tag ver­ab­schie­det IT-Si­cher­heits­ge­setz
		https://www.bmi.bund.de/SharedDocs/Pressemitteilungen/DE/2015/06/bundestag-beschliesst-it-sicherheitsgesetz.html
		10.07.2015  
		IT-Sicherheitsgesetz hat Bundesrat passiert – Papiertiger ist verabschiedet
		https://netzpolitik.org/2015/it-sicherheitsgesetz-hat-bundesrat-passiert-papiertiger-ist-verabschiedet/
\end{comment}


\end{document}







